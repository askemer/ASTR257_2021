\documentclass[12pt]{article}
\usepackage{epsfig}
\textwidth 7.5in                     % page width in inches
\textheight 9.9in                    % page height in inches
\topmargin -90pt                     % to fit on A4
\oddsidemargin -36pt                 % Left hand margin (odd pages)
\evensidemargin -28pt
\baselineskip 0.168in
\setlength{\parindent}{20pt}
%\def\etal{{\it et al. }}
\def\etal{et al.\ }
\newcommand{\msun}{\hbox{M$_{\odot}$}}

\begin{document}

\noindent
\centerline{\bf \textit{Nickel Observations with UCSC ASTR 257: Modern Astronomical Techniques}} 
\centerline{\bf PI A.\ Skemer}

\vskip 15pt

\centerline{\bf  Scientific Justification: }

As part of UC Santa Cruz Astronomy's graduate curriculum, Andy Skemer and Xavier Prochaska teach an observing class, which consists of a week-long field trip to Lick Observatory.  The class is required of all first-year graduate students, and ensures that (1) our students will develop a broad range of observational skills at a time when opportunities to develop these skills are becoming more rare, and (2) our students will develop experience with Lick Observatory facilities, which will incentives them to become immediate scientific users.  This class has some parallels to the graduate observing workshop that has been run for many years at Lick Observatory, but the formal course will be more time intensive and will fulfill specific department requirements.

2019 was the first time we offered this class.  It was a huge success, with eleven students completing a class where they used Lick facilities to do astrometry, photometry, spectroscopy, and adaptive optics.  A publicly accessible github account with the class schedule, lectures, observing activities and raw data is available at https://github.com/askemer/ASTR257.  The class was canceled in 2020 due to COVID.

We are planning two observations with the Nickel 1-meter.  We will observe Pluto over two nights to determine its proper motion and we will observe an open cluster to make an HR diagram.  Both observations are mainstays of the PHYS/ASTR 136 undergraduate course and ASTR 257.  We request two consecutive first-half nights for this program.
\\\\
\textbf{Proposed schedule for ASTR 257}\\
Saturday, 9/18—Arrive and Tour\\
Sunday, 9/19—Observe with Nickel (1st half)\\
Monday, 9/20—Observe with Nickel (1st half)\\
Tuesday, 9/21—Observe with Refractor (1st half)\\
Wednesday, 9/22—Observe with NGS-AO (1st half)\\
Thursday, 9/23—Observe with KAST (1st half)\\
Friday, 9/24—Depart

\newpage

\end{document}